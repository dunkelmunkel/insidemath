\documentclass{../cssheet}

%--------------------------------------------------------------------------------------------------------------
% Basic meta data
%--------------------------------------------------------------------------------------------------------------

\title{Zahlen haben einen hohen Stellenwert}
\author{Prof. Dr. Christian Spannagel}
\date{\today}
\hypersetup{%
    pdfauthor={\theauthor},%
    pdftitle={\thetitle},%
    pdfsubject={Aufgabenblatt Inside Math!},%
    pdfkeywords={insidemath}
}

%--------------------------------------------------------------------------------------------------------------
% document
%--------------------------------------------------------------------------------------------------------------

\begin{document}
\printtitle

\textbf{Aufgabe 1 (Wie im alten Ägypten und Rom):}  Stellt die Zahlen $74$, $440$ und $2023$ so dar, wie es die alten Ägypter*innen und Römer*innen getan haben.

\textbf{Aufgabe 2 (Eier im b-Pack):} Stellt die folgenden Zahlen in anderen Stellenwertsystemen dar:
\begin{enumerate}[a)]
\item $197$ im 4er-System
\item $211$ im 5er-System
\item $255$ im 2er-System
\end{enumerate}
Überlegt euch auch weitere Beispiele! Wie könnt ihr eure Ergebnisse überprüfen?
Warum funktioniert das Verfahren eigentlich? Besprecht das untereinander!

\textbf{Aufgabe 3 (Das brauch ich schriftlich):}  Schriftliches Rechnen in anderen Stellenwertsystemen ist gar nicht so schwer, wenn man den Kniff raus hat. Macht euch zunächst einmal klar, wie man im Dezimalsystem schriftlich addiert, subtrahiert, multipiziert und dividiert. Rechnet anschließend die folgenden Aufgaben schriftlich im Sechsersystem. Wichtig: Ihr dürft die Zahlen nicht ins Dezimalsystem umrechnen und dann addieren usw., sondern ihr sollt das Sechsersystem nicht verlassen! (Selbstverständlich dürft ihr anschließend alle Zahlen einmal in Dezimalsystem umrechnen um zu prüfen, ob ihr richtig gerechnet habt. Rechnen müsst ihr aber im Sechsersystem.)

\begin{enumerate}[a)]
\item $[444153]_6 + [113242]_6$
\item $[5325]_6 - [2342]_6$
\item $[452]_6 \cdot [24]_6$
\item $[22304]_6 : [5]_6$
\end{enumerate}

Stellt euch außerdem gegenseitig Aufgaben in anderen Stellenwertsystemen (Basis $2$, $3$, $4$, $5$, $6$, \ldots).

\textbf{Lösungshinweise:} 

Diese Lösungshinweise helfen euch zu schauen, ob ihr richtig gerechnet habt. Falls ihr Fragen zum Lösungsweg habt oder euch die Lösung nicht klar ist, dann postet sie in Discord oder in Moodle!

\begin{itemize}
\item Aufgabe~2: 3011; 1321; 11111111 (Hättet ihr das letzte Ergebnis auch schneller erhalten können?)
\item Aufgabe~3: 1001435; 2543; 21012; 2523 Rest 1
\end{itemize}

\vspace*{10mm}
\printlicense

\printsocials

\end{document}
