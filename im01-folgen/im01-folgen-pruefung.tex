\documentclass{cssheet}


%--------------------------------------------------------------------------------------------------------------
% Basic meta data
%--------------------------------------------------------------------------------------------------------------

\title{Folgenreiche Prüfungsaufgaben}
\author{Prof. Dr. Christian Spannagel}
\date{\today}

%--------------------------------------------------------------------------------------------------------------
% document
%--------------------------------------------------------------------------------------------------------------

\begin{document}
\printtitle

\begin{aufgabe}[WiSe 16/17]
	Gegeben sei die Zahlenfolge: $5,10,20,40,80,160, \dots$
	\\
	Geben Sie für diese Folge Formeln in rekursiver und geschlossener Form an.
\end{aufgabe}

\begin{aufgabe}[SoSe 17]
	Gegeben sei die Zahlenfolge: $3, 15, 75, 375, 1875, \dots$
	\\
	Geben Sie für diese Folge Formeln in rekursiver und geschlossener Form an.
\end{aufgabe}

\begin{aufgabe}[SoSe 22]
	Gegeben sei die Zahlenfolge: $4,-12,36,-108,324, \dots$
	\\
	Geben Sie für diese Folge Formeln in rekursiver und expliziter Form an.
\end{aufgabe}

\begin{aufgabe}[SoSe 23]
	Gegeben sei die Zahlenfolge: $4,-12,36,-108,324, \dots$
	\\
	Gib für diese Folge Formeln in rekursiver und geschlossener Form an.
\end{aufgabe}

\begin{aufgabe}[WiSe 23/24]
	Gegeben sei die Zahlenfolge: $2,-6,18,-54,162, \dots$
	\\
	Gib für diese Folge Formeln in rekursiver und geschlossener Form an.
\end{aufgabe}

\begin{aufgabe}[SoSe 24]
	Gegeben sei die Zahlenfolge: $3,6,12,24,48, \dots$
	\\
	Gib für diese Folge jeweils eine Formel in rekursiver und geschlossener (expliziter) Form an.
\end{aufgabe}

\begin{aufgabe}[WiSe 24/25]
	Gegeben sei die Zahlenfolge: $4,-12,36,-108,324, \dots$
	\\
	Gib die Zahlenfolge in rekursiver und in geschlossener (expliziter) Form an.
\end{aufgabe}


\vspace*{10mm}
\printlicense

\printsocials



\end{document}
