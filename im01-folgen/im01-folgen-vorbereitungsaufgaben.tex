\documentclass{../cssheet}


%--------------------------------------------------------------------------------------------------------------
% Basic meta data
%--------------------------------------------------------------------------------------------------------------

\title{Folgenreiche Aufgaben}
\author{Prof. Dr. Christian Spannagel}
\date{\today}
\hypersetup{%
    pdfauthor={\theauthor},%
    pdftitle={\thetitle},%
    pdfsubject={Aufgabenblatt Inside Math!},%
    pdfkeywords={insidemath}
}

%--------------------------------------------------------------------------------------------------------------
% document
%--------------------------------------------------------------------------------------------------------------

\begin{document}
\printtitle

\vspace*{10mm}

\textbf{Vorbemerkungen:} 
\emph{Es gelten die gleichen Vorbemerkungen wie beim letzten Aufgabenblatt. :-)}

\textbf{Aufgabe 1 (Wie geht es weiter?):}  Setzt die Zahlenfolgen fort. Vergleicht eure Ergebnisse mit
den Ergebnissen anderer Student*innen. Gibt es auch mehrere Möglichkeiten?
\begin{enumerate}[a)]
\item $2, 4, 6, 8, \ldots$
\item $1, 2, 4, 8, \ldots$
\item $1, 1, 2, 3, 5, 8, \ldots$
\item $4, 5, 2, 7, 0, \ldots$
\item $7, 9, 6, 18, 20, 17, 51, \ldots$
\item $3, 5, 7, \ldots$
\item $3, 1, 4, 1, 5, \ldots$
\end{enumerate}



\textbf{Aufgabe 2 (Wie tief geht's rekursiv?):}  Findet die Formeln für die Zahlenfolgen jeweils in der
rekursiven und in der geschlossenen (expliziten) Form. 

\begin{enumerate}[a)]
\item $3, 6, 9, 12, 15, \ldots$
\item $2, 8, 32, 128, 512, \ldots$
\item $1, -3, 9, -27, \ldots$
\item $3, 6, 11, 18, 27, 38, \ldots$
\end{enumerate}

Überlegt euch selbst weitere interessante Folgen und postet sie ins Forum. Mal sehen, wie schnell die anderen die rekursiven und expliziten Darstellungen für eure Folgen finden :).

\textbf{Aufgabe 3 (Von Bienchen und Blümchen):} 

Die folgenden Folgen (haha) wurden nach den Fibonacci-Regeln entworfen, aber einige Folgenglieder wurden weggelassen. Welche Zahlen fehlen?
\begin{enumerate}[a)]
\item $\_\  \_\  50\  \_\  127\  \_$
\item $\_\  \_\  72\ \ 104\  \_$
\item $\_\  \_\  53\  \_\  126$
\item $\_\ 15\  \_\  \_\  69$
\end{enumerate}

%\vspace*{10mm}
\newpage
\printlicense

\printsocials


\end{document}
