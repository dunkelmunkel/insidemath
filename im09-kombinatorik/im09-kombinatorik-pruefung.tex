\documentclass{cssheet}


%--------------------------------------------------------------------------------------------------------------
% Basic meta data
%--------------------------------------------------------------------------------------------------------------

\title{Prüfungsaufgaben zählen}
\author{Prof. Dr. Christian Spannagel}
\date{\today}

%--------------------------------------------------------------------------------------------------------------
% document
%--------------------------------------------------------------------------------------------------------------

\begin{document}
\printtitle


\begin{aufgabe}[SoSe 17, HT]
	Es werden Steckwürfeltürme mit drei Stockwerken gebaut. Es stehen hierfür die Farben blau, rot, lila und grün zur Verfügung. Wie viele verschiedene Steckwürfeltürme kann man bauen, wenn
	\begin{enumerate}
		\item \dots Farben auch mehrfach verwendet werden dürfen?
		\item \dots Farben nicht mehrfach verwendet werden dürfen?
	\end{enumerate}
	
	\textit{Hinweis: Es kommt bei den Steckwürfeltürmen auf die Reihenfolge an!}
\end{aufgabe}

\begin{aufgabe}[SoSe 17, NT]
	Es werden Steckwürfeltürme mit vier Stockwerken gebaut. Es stehen hierfür die Farben blau, rot, gelb, lila und grün zur Verfügung. Wie viele verschiedene Steckwürfeltürme kann man bauen, wenn
	\begin{enumerate}
		\item \dots Farben auch mehrfach verwendet werden dürfen?
		\item \dots Farben nicht mehrfach verwendet werden dürfen?
	\end{enumerate}
	
	\textit{Hinweis: Es kommt bei den Steckwürfeltürmen auf die Reihenfolge an!}
\end{aufgabe}

\begin{aufgabe}[SoSe 22]
	Tim besitzt $12$ Bücher und möchte sich davon $4$ Bücher auf eine Bahnfahrt mitnehmen.
	\begin{enumerate}
		\item Wie viele verschiedene Möglichkeiten hat er?
		\item Tim hat sich für $4$ Bücher entschieden und wählt die Reihenfolge, in der er sie lesen möchte. Wie viele Möglichkeiten gibt es?
		\item Tim hat sich für $4$ Bücher entschieden. Er liest gerne Bücher auch mehrfach und schafft es, auf der Zugfahrt $3$ Bücher zu lesen. Tims Mutter will wissen, welche Bücher er in welcher Reihenfolge gelesen hat. Wie viele Möglichkeiten gibt es?
	\end{enumerate}
\end{aufgabe}

\begin{aufgabe}[SoSe 22]
	Tim besitzt $12$ Bücher und möchte sich davon $4$ Bücher auf eine Bahnfahrt mitnehmen.
	\begin{enumerate}
		\item Wie viele verschiedene Möglichkeiten hat er?
		\item Tim hat sich für $4$ Bücher entschieden und wählt die Reihenfolge, in der er sie lesen möchte. Wie viele Möglichkeiten gibt es?
	\end{enumerate}
\end{aufgabe}

\begin{aufgabe}[WiSe 23/24]
	\begin{aufgabe}[SoSe 17, NT]
		Du hast Steckwürfel in vier Farben (rot, grün, blau, orange).
		Du baust Steckwürfeltürme mit drei Stockwerken. Wie viele Möglichkeiten gibt es, wenn
		\begin{enumerate}
			\item \dots die Reihenfolge relevant ist und du Farben auch mehrfach verwenden kannst?
			\item \dots die Reihenfolge relevant ist und jede Farbe höchstens einmal verwendet werden darf?
			\item \dots die Reihenfolge irrelevant ist und jede Farbe höchstens einmal verwendet werden darf?
		\end{enumerate}
	\end{aufgabe}
\end{aufgabe}

\begin{aufgabe}[SoSe 24]
	Wie viele Möglichkeiten gibt es jeweils?
	\begin{enumerate}
		\item Du hast $4$ T-Shirts, $2$ Hosen und $3$ Jacken, die sich alle jeweils voneinander unterscheiden. Auf wie viele Arten kannst du dich kleiden?
		\item Du hast rote, schwarze, blaue und grüne Socken. Auf wie viele Arten kannst du die Socken anziehen, wenn du auf deinen linken und rechten Fuß sowohl gleichfarbige als auch unterschiedliche Socken anziehen kannst?
		\item Du kombinierst die Möglichkeiten von Aufgabe a) und Aufgabe b). Auf wie viele Arten kannst du dich kleiden?
	\end{enumerate}
\end{aufgabe}

\begin{aufgabe}[WiSe 24/25]
	Du besitzt 15 Bücher und möchtest davon 4 Bücher mit auf eine Reise nehmen. Wie viele Möglichkeiten gibt es?
\end{aufgabe}


\vspace*{10mm}
\printlicense

\printsocials



\end{document}
