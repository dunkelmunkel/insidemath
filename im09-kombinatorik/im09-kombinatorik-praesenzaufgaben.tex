\documentclass{../cssheet}

%--------------------------------------------------------------------------------------------------------------
% Basic meta data
%--------------------------------------------------------------------------------------------------------------

\title{Kombinatorik -- Präsenzaufgaben}
\author{Prof. Dr. Christian Spannagel}
\date{\today}
\hypersetup{%
    pdfauthor={\theauthor},%
    pdftitle={\thetitle},%
    pdfsubject={Aufgabenblatt Inside Math!},%
    pdfkeywords={insidemath,kombinatorik}
}

%--------------------------------------------------------------------------------------------------------------
% document
%--------------------------------------------------------------------------------------------------------------

\begin{document}
\printtitle

\vspace*{10mm}


\textbf{Aufgabe 1 (Kombinatorik in der Eisdiele):}  In einer Eisdiele gibt es 10 Sorten Eis. Ihr möchtet eine Waffel mit drei Kugeln kaufen. Wie viele Möglichkeiten gibt es, wenn\ldots
\begin{enumerate}[a)]
\item \ldots Eissorten auch mehrfach vorkommen können, es aber relevant ist, in welcher Reihenfolge sie auf der Waffel sind.
\item \ldots Eissorten nur einmal vorkommen dürfen, die Reihenfolge aber weiterhin relevant ist.
\item \ldots Eissorten nur einmal vorkommen dürfen, aber egal in welcher Reihenfolge.
\end{enumerate}
Tipp: Sucht nicht gleich nach irgendwelchen Formeln, sondern visualisiert euch die Situation und überlegt euch daran die Lösung.


\vspace*{3cm}

\printlicense

\printsocials



\end{document}
