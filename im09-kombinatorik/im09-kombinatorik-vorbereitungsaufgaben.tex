\documentclass{../cssheet}
%--------------------------------------------------------------------------------------------------------------
% Basic meta data
%--------------------------------------------------------------------------------------------------------------

\title{Die Kunst des Zählens}
\author{Prof. Dr. Christian Spannagel}
\date{\today}
\hypersetup{%
    pdfauthor={\theauthor},%
    pdftitle={\thetitle},%
    pdfsubject={Aufgabenblatt Inside Math!},%
    pdfkeywords={insidemath,kombinatorik}
}

%--------------------------------------------------------------------------------------------------------------
% document
%--------------------------------------------------------------------------------------------------------------

\begin{document}
\printtitle

\vspace*{10mm}

\textbf{Aufgabe 1 (Die zwei Türme):}  Aus roten, blauen, grünen und schwarzen Legosteinen sollen möglichst viele verschiedenen Türme mit zwei Etagen gebaut werden.
\begin{enumerate}[a)]
\item In jedem Turm sollen zwei unterschiedliche Farben vorkommen. Wie viele verschiedene Türme gibt es? (Die Anordnung der Farben ist von Relevanz).
\item Wie viele Türme gibt es, wenn die Farben auch mehrfach verwendet werden dürfen?
\item In jedem Turm sollen zwei unterschiedliche Farben vorkommen. Diesmal geht es aber nur um die unterschiedlichen Farben der Legosteine, nicht um deren Anordnung. Die Anordnung der Farbe ist also unbedeutend.
\item Wie viele Türme gibt es, wenn die Anordnung der Farben egal ist (wie bei c), aber die Farben mehrfach verwendet werden dürfen? 
\item Aus roten, blauen und grünen Legosteinen sollen möglichst viele verschiedene Türme mit drei Etagen gebaut werden. In jedem Turm soll jede der drei Farben vorkommen. Wie viele verschiedene Türme gibt es?
\end{enumerate}
Tipp: Sucht nicht gleich nach irgendwelchen Formeln, sondern visualisiert euch die Situation und überlegt euch daran die Lösung.


\textbf{Aufgabe 2 (Emil und die Detektive):} Ein fünfstelliges Zahlenschloss soll geknackt werden. Wie viele verschiedene Zahlenkombinationen existieren, wenn an jeder Stelle sieben verschiedene Ziffern eingestellt werden können? 


\textbf{Aufgabe 3 (Harry Potter):} 
\begin{enumerate}[a)]
\item Ihr besitzt alle sieben Harry-Potter-Bücher und ihr möchtet auf eine Reise drei davon mitnehmen. Wie viele Möglichkeiten gibt es?
\item Ihr habt euch für drei Bücher entschieden. In wie vielen verschiedenen Reihenfolgen könnt ihr sie lesen?
\end{enumerate}


\textbf{Aufgabe 4 (Hans im Glück):} Bei einem Glücksspiel werden aus 49 Kugeln, die mit den Nummern 1 bis 49 beschriftet sind, 6 Gewinnkugeln gezogen. Wie viele Tippmöglichkeiten gibt es insgesamt? 

\textbf{Aufgabe 5 (Die Geschichte vom Suppenkasper):} Wie viele Menüs kann man zusammenstellen, wenn man zwei Vorspeisensuppen hat, drei Hauptspeisen und zwei Desserts?


\textbf{Lösungshinweise:} 

Diese Lösungshinweise helfen euch zu schauen, ob ihr richtig gerechnet habt. Falls ihr Fragen zum Lösungsweg habt oder euch die Lösung nicht klar ist, dann postet sie in Discord oder in Moodle!

\begin{itemize}
\item Aufgabe~1: 12, 16, 6, 10, 6
\item Aufgabe~2: 16.807
\item Aufgabe~3: 35, 6
\item Aufgabe~4: 13.983.816
\item Aufgabe~5: 12
\end{itemize}

\newpage
%\vspace*{3cm}
\printlicense

\printsocials

\end{document}
