\documentclass{cssheet}


%--------------------------------------------------------------------------------------------------------------
% Basic meta data
%--------------------------------------------------------------------------------------------------------------

\title{\Large Geteilte Prüfungsaufgaben sind halbe Prüfungsaufgaben}
\author{Prof. Dr. Christian Spannagel}
\date{\today}

%--------------------------------------------------------------------------------------------------------------
% document
%--------------------------------------------------------------------------------------------------------------

\begin{document}
\printtitle

\begin{aufgabe}[WiSe 16/17]
	Wie lautet die Teilbarkeitsregel bezüglich der Division durch $8$? Begründen Sie
	diese!
\end{aufgabe}

\begin{aufgabe}[SoSe 17]
	\begin{enumerate}
		\item Geben Sie die Teilermengen der Zahlen $12$, $16$ und $18$ an.
		\item Stellen Sie das Verhältnis dieser drei Mengen zueinander in einem Venn-Diagramm dar.
		\item Geben Sie die folgende Menge an: $(T(16) \cap T(18)) \backslash T(12)$
	\end{enumerate}
\end{aufgabe}

\begin{aufgabe}[SoSe 22]
	Teilbarkeit durch die Zahl 8:
	\begin{enumerate}
		\item Geben Sie die Regel für die Teilbarkeit durch $8$ an.
		\item Begründen Sie diese Teilbarkeitsregel mit Hilfe des Gummibärchenmodells.
		\item Ist die Zahl $0$ durch $8$ teilbar? Begründen Sie!
	\end{enumerate}
\end{aufgabe}

\begin{aufgabe}[SoSe 22]
	Geben Sie alle Teiler der Zahlen $180$ und $75$ an.
\end{aufgabe}

\begin{aufgabe}[SoSe 23]
	Teilbarkeitsregeln:
	\begin{enumerate}
		\item Gib die Regeln für die Teilbarkeit durch $3$, $4$, $8$ und $9$ an.
		\item Begründe die Regel für die Teilbarkeit durch $3$ mit Hilfe des Gummibärchenmodells.
	\end{enumerate}
\end{aufgabe}

\begin{aufgabe}[WiSe 23/24]
	Wie lautet die Teilbarkeitsregel für die Teilbarkeit durch $3$? Begründe die Regel mit dem Gummibärchenmodell.
\end{aufgabe}

\begin{aufgabe}[WiSe 23/24]
	Gib die Primfaktorzerlegung der Zahl 2024 an. Zeichne das Hassediagramm der Zahl 2024.
\end{aufgabe}

\begin{aufgabe}[SoSe 24]
	Wie lautet die Teilbarkeitsregel für die Teilbarkeit durch $3$? Begründe die Regel mit dem Gummibärchenmodell.
\end{aufgabe}

\begin{aufgabe}[SoSe 24]
	Wie viele Teiler hat die Zahl $294$? Gib alle Teiler der Zahl $294$ an!
\end{aufgabe}

\begin{aufgabe}[WiSe 24/25]
	Wie lautet die Primfaktorzerlegung der Zahl $2025$? Zeichne das Hassediagramm der Zahl 2025. Beschrifte das Hassediagramm so, dass man alle Teiler der Zahl $2025$ daran ablesen kann. Rechne dabei alle Zahlen aus.
\end{aufgabe}


\vspace*{10mm}
\printlicense

\printsocials



\end{document}
