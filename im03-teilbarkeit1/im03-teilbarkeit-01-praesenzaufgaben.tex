\documentclass{../cssheet}

%--------------------------------------------------------------------------------------------------------------
% Basic meta data
%--------------------------------------------------------------------------------------------------------------

\title{Teilen ist schön! (Präsenzaufgaben)}
\author{Prof. Dr. Christian Spannagel}
\date{\today}
\hypersetup{%
    pdfauthor={\theauthor},%
    pdftitle={\thetitle},%
    pdfsubject={Aufgabenblatt Inside Math!},%
    pdfkeywords={insidemath}
}


%--------------------------------------------------------------------------------------------------------------
% document
%--------------------------------------------------------------------------------------------------------------

\begin{document}
\printtitle

\vspace*{10mm}

\textbf{Aufgabe 1 (Bauklötze):}  Ihr wollt 96 Bauklötze gleichmäßig auf Türme aufteilen.
\begin{enumerate}[a)]
	\item Wie viele verschiedene Möglichkeiten gibt es, gleich hohe Türme zu bauen? Hinweis: Die Bauklötze in einem Turm dürfen nur übereinander, nicht nebeneinander gesetzt werden.
	\item Ändert sich die Anzahl der Möglichkeiten, wenn ihr 97 Bauklötze habt?
\end{enumerate}

\textbf{Aufgabe 2 (Ungerade Teileranzahl):}  Findet Zahlen, die eine ungerade Anzahl an Teilern haben. Um welche Zahlen dreht es sich?

\textbf{Aufgabe 3 (Beziffert das mal!):}  Bildet fünfstellige Zahlen aus unterschiedlichen (!) Ziffern, die teilbar sind:
\begin{enumerate}[a)]
\item durch 15
\item durch 3, aber nicht durch 9
\item durch 3 und 8
\item durch 2, aber nicht durch 4
\end{enumerate}

\textbf{Aufgabe 4 (Manchmal klappt's, manchmal nicht!):} Die folgende Aussage ist korrekt: Wenn eine Zahl durch $2$ und durch $3$ teilbar ist, dann ist sie auch durch $6$ teilbar. Unter welcher Voraussetzung ist die folgende allgemeine Aussage korrekt: Wenn eine Zahl durch $a$ und durch $b$ teilbar ist, dann ist sie auch durch $a\cdot b$ teilbar.
Sucht Beispiele und Gegenbeispiele und findet die Regel!

\textbf{Aufgabe 5 (Alle Gummibärchen für mich!):}  

Begründet die Teilbarkeitsregel für die Zahl $4$ und $8$ mit Hilfe des Gummibärchenmodells. (Achtung: Hier gilt keine Quersummenregel!)


\vspace*{10mm}
\printlicense

\printsocials

\end{document}
