\documentclass{../cssheet}

%--------------------------------------------------------------------------------------------------------------
% Basic meta data
%--------------------------------------------------------------------------------------------------------------

\title{Aufgaben für Entdecker*innen (Teil 2)}
\author{Prof. Dr. Christian Spannagel}
\date{\today}
\hypersetup{%
    pdfauthor={\theauthor},%
    pdftitle={\thetitle},%
    pdfsubject={Aufgabenblatt Inside Math!},%
    pdfkeywords={insidemath}
}

%--------------------------------------------------------------------------------------------------------------
% document
%--------------------------------------------------------------------------------------------------------------

\begin{document}
\printtitle

\textbf{Vorbemerkungen:} 
\emph{Es gelten die gleichen Vorbemerkungen wie beim letzten Aufgabenblatt.}


\textbf{Aufgabe 1 (Partytime):}  Auf einer Party sind $100$ Personen und stoßen mit ihren Sektkelchen an. Jeder stößt natürlich mit jedem genau einmal an. Wie oft hört man Kelche klingen? (Wie kannst du dich am besten an die Lösung „heranpirschen“?) 

\textbf{Aufgabe 2 (Figurbetont):}  Zeichnet ein Dreieck, Viereck, Fünfeck, Sechseck usw. Zeichnet jeweils alle Diagonalen. Bestimmt für jedes Vieleck die Anzahl der Seiten und der Diagonalen. Bestimmt dann die Anzahl der insgesamt gezeichneten Strecken (Seiten $+$ Diagonalen) in dem jeweiligen Vieleck. Wie könnt ihr eure Ergebnisse übersichtlich darstellen? Welche Regelmäßigkeiten entdeckt ihr? 

\textbf{Aufgabe 3 (Ich spring im Quadrat):} Addiert „von vorne“ die ersten ungeraden natürlichen Zahlen ($1$, $3$, $5$, $7$, …). Welche Ergebnisse erhaltet ihr, wenn ihr immer eine weitere Zahl dazu nehmt? Was könnt ihr entdecken? Wie könnt ihr begründen, dass eure gefundene Regelmäßigkeit stimmt? 

\vspace*{10mm}

\printlicense

\printsocials


\end{document}
