\documentclass{../cssheet}

%--------------------------------------------------------------------------------------------------------------
% Basic meta data
%--------------------------------------------------------------------------------------------------------------

\title{Eine Menge Spaß!}
\author{Prof. Dr. Christian Spannagel}
\date{\today}
\hypersetup{%
    pdfauthor={\theauthor},%
    pdftitle={\thetitle},%
    pdfsubject={Aufgabenblatt Inside Math!},%
    pdfkeywords={insidemath}
}


%--------------------------------------------------------------------------------------------------------------
% document
%--------------------------------------------------------------------------------------------------------------

\begin{document}
\printtitle

\vspace*{10mm}

\textbf{Aufgabe 1 (Jede Menge Mengen):}  Aus der Schule kennt ihr jede Menge unendliche
Mengen. Erinnert ihr euch? Mit welchen Mengen kamt ihr in der Schule in Berührung? Wie
stehen diese Mengen in Beziehung zueinander?

\textbf{Aufgabe 2 (Aus dem Bauch heraus):} Versucht einmal intuitiv den Begriff „Menge“ zu
definieren. Was genau ist eine Menge?

\textbf{Aufgabe 3 (Mengenoperationen):}  In dem Video zu den Mengenoperationen werden noch
einige Definitionen verlangt. Nehmen wir mal an, $A$ und $B$ sind zwei Mengen. Versucht
einmal, \emph{Schnittmenge von A und B}, \emph{Differenzmenge von A und B} und
\emph{Komplementärmenge von A in B} zu definieren. Ihr müsst dabei nicht so formal vorgehen
wie in dem Video, in Worten reicht es. Nehmen wir nochmal das Beispiel
\emph{Vereinigungsmenge von A und B}:

\begin{quote}
Die Vereinigungsmenge von $A$ und $B$ ist diejenige Menge, deren Elemente in $A$ oder in $B$ enthalten sind. Man schreibt $A \cup B$.
\end{quote}

Jetzt seid ihr dran!
 
\textbf{Aufgabe 4 (Mut zur Lücke):}  
Vervollständigt:

\begin{enumerate}[a)]
\item $A \cap A = $
\item $A \cup A = $
\item $A\setminus A = $
\item $A\cap \emptyset =$
\item $A\cup \emptyset =$
\item $A\cap B = \emptyset$, dann könnte $A=$ \hspace{1cm} und $B=$ \hspace{1cm} sein.
\end{enumerate}

\textbf{Aufgabe 5 (Wenn das Wörtchen Venn nicht wär):}  
Beweist oder widerlegt mit Venn-Diagrammen:
\begin{enumerate}[a)]
\item $(A \cap B) \cap C = A \cap (B \cap C)$
\item $A \cap (B \cup C) = (A \cap B) \cup (A \cap  C)$
\item $C\setminus (A\cap B) = (C\setminus A) \cup (C\setminus B)$
\item $(A\cup B)\setminus C = (A\setminus C) \cup (B\setminus C)$
\item $(A\cap B)\cup C = (A\cap C) \cup (B\cap C)$
\item $A\cup B = A \cup (B\setminus A)$
\item $(A\cap B)\setminus C = (A\setminus C) \cup (B\setminus C)$
\end{enumerate}

\newpage

\textbf{Lösungshinweise:} 

Diese Lösungshinweise helfen euch zu schauen, ob ihr richtig gerechnet habt. Falls ihr Fragen zum Lösungsweg habt oder euch die Lösung nicht klar ist, dann postet sie in Discord oder in Moodle!

\begin{itemize}
\item Bei Aufgabe~5 sind zwei Aussagen falsch. 
\end{itemize}

\vspace*{10mm}
\printlicense

\printsocials
\end{document}
