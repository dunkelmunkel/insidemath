\documentclass{cssheet}


%--------------------------------------------------------------------------------------------------------------
% Basic meta data
%--------------------------------------------------------------------------------------------------------------

\title{Eine Menge Prüfungsaufgaben}
\author{Prof. Dr. Christian Spannagel}
\date{\today}

%--------------------------------------------------------------------------------------------------------------
% document
%--------------------------------------------------------------------------------------------------------------

\begin{document}
\printtitle

\begin{aufgabe}[WiSe 16/17]
	Beweisen oder widerlegen Sie mit Venn-Diagrammen:
	$
	(A \cup B) \cap(B \cup C)=B \cup(A \cap C)
	$
\end{aufgabe}

\begin{aufgabe}[SoSe 22]
	Beweisen oder widerlegen Sie mit Venn-Diagrammen:
	$
	(A \cap B) \backslash C=(A \backslash C) \cap(B \backslash C)
	$
\end{aufgabe}

\begin{aufgabe}[SoSe 23]
	Beweisen oder widerlegen Sie mit Venn-Diagrammen:
	$
	A \cup (B \cap C)=(A \cup B) \cap(A \cup C)
	$
\end{aufgabe}

\begin{aufgabe}[WiSe 23/24]
	Beweise oder widerlege mit einem Venn-Diagramm:
	$A \cap(B \backslash C)=(A \cap B) \backslash C$
\end{aufgabe}

\begin{aufgabe}[WiSe 24/25]
	Beweise oder widerlege mit einem Venn-Diagramm:
	$
	(A \cap B) \cap C=A \cap(B \cap C)
	$
\end{aufgabe}

\begin{aufgabe}[angelehnt an SoSe 25]
	Zeichne ein Mengendiagramm mit den folgenden Mengen:
	\begin{itemize}
		\item $M_1=$ Menge aller Dreiecke
		\item $M_2=$ Menge aller gleichseitigen Dreiecke
		\item $M_3=$ Menge aller gleichschenkligen Dreiecke
		\item $M_4=$ Menge aller Vierecke
	\end{itemize}
\end{aufgabe}


\vspace*{10mm}
\printlicense

\printsocials



\end{document}
