\documentclass{cssheet}


%--------------------------------------------------------------------------------------------------------------
% Basic meta data
%--------------------------------------------------------------------------------------------------------------

\title{Geometrische Prüfungsaufgaben}
\author{Prof. Dr. Christian Spannagel}
\date{\today}

%--------------------------------------------------------------------------------------------------------------
% document
%--------------------------------------------------------------------------------------------------------------

\begin{document}
\printtitle

\begin{aufgabe}[WiSe 16/17]
	Was ergibt die Verkettung von zwei Achsenspiegelungen an zwei parallelen Geraden? Erläutern Sie dies an einer geeigneten Zeichnung.
\end{aufgabe}

\begin{aufgabe}[SoSe 17, HT]
	\begin{enumerate}
		\item Skizzieren Sie eine Schubspiegelung.
		\item Ist die Schubspiegelung orientierungstreu? Begründen Sie!
		\item Hat die Schubspiegelung Fixpunkte, Fixgeraden und/oder Fixpunktgeraden? Begründen Sie!
	\end{enumerate}
\end{aufgabe}

\begin{aufgabe}[SoSe 17, NT]
	\begin{enumerate}
		\item Skizzieren Sie eine Punktspiegelung.
		\item Ist die Punktspiegelung orientierungstreu? Begründen Sie!
		\item Hat die Punktspiegelung Fixpunkte, Fixgeraden und/oder Fixpunktgeraden? Begründen Sie!
	\end{enumerate}
\end{aufgabe}

\begin{aufgabe}[SoSe 22]
	Eine Schubspiegelung lässt sich durch die Nacheinanderausführung von drei Achsenspiegelungen erzeugen.
	\begin{enumerate}
		
		\item Zeichnen Sie eine passende Skizze, in der ein Dreieck schubgespiegelt wird.
		\item Hat die Schubspiegelung Fixpunkte, Fixgeraden und Fixpunktgeraden? Begründen Sie!
		\item Ist die Schubspiegelung längentreu, parallelentreu und orientierungstreu? Begründen Sie!
	\end{enumerate}
\end{aufgabe}

\begin{aufgabe}[WiSe 23/24]
	Eine Punktspiegelung lässt sich durch die Nacheinanderausführung von zwei Achsenspiegelungen darstellen.
	\begin{enumerate}
		\item Zeichne eine passende Skizze, in der ein Dreieck punktgespiegelt wird.
		\item Hat die Punktspiegelung Fixpunkte, Fixgeraden oder Fixpunktgeraden? Begründe!
		\item Ist die Punktspiegelung orientierungstreu? Begründe!
	\end{enumerate}
\end{aufgabe}


\vspace*{10mm}
\printlicense

\printsocials



\end{document}
