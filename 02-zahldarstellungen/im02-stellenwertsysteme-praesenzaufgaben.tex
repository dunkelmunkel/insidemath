\documentclass{../cssheet}

%--------------------------------------------------------------------------------------------------------------
% Basic meta data
%--------------------------------------------------------------------------------------------------------------

\title{Noch mehr Aufgaben mit Stellenwert}
\author{Prof. Dr. Christian Spannagel}
\date{\today}
\hypersetup{%
    pdfauthor={\theauthor},%
    pdftitle={\thetitle},%
    pdfsubject={Aufgabenblatt Inside Math!},%
    pdfkeywords={insidemath}
}

%--------------------------------------------------------------------------------------------------------------
% document
%--------------------------------------------------------------------------------------------------------------

\begin{document}
\printtitle


\textbf{Aufgabe 1 (Just a little bit):}  In der Informatik bezeichnet man eine Informationseinheit, die zwei Zustände ($0$ und $1$ bzw. \emph{kein Strom} und \emph{Strom}) einnehmen kann, als \emph{Bit}. Wir bewegen uns somit im Binärsystem. 
\begin{enumerate}[a)]
\item Stellt die Zahlen $16$, $31$ und $48$ im Binärsystem einmal mit 4~Bit, einmal mit 8~Bit dar. 
\item Wie viele verschiedene Zahlen kann man mit $4$~Bit ($8$~Bit, $16$~Bit, $32$~Bit, $64$~Bit) darstellen? Was ist die jeweils größte darstellbare Zahl?  (Wenn die Zahlen sehr groß werden, dürft ihr auch die Potenzschreibweise verwenden.)
\end{enumerate}

\textbf{Aufgabe 2 (Bitte 8 Bit!):} 
\begin{enumerate}[a)]
\item Was passiert, wenn man eine im Binärsystem dargestellte Zahl mit $2$ multipliziert? Was, wenn man sie durch $2$ dividiert? Findet ein einfaches Verfahren, indem ihr es zunächst an ein paar Beispielen ausprobiert.
\item Addiert $171$ und $255$ im Binärsystem. Wie geht das sehr einfach? Findet ein Verfahren!
\item Denkt euch eine Binärzahl mit 8~Bit aus. Schreibt dann eine Zahl auf, bei der im Vergleich zur ersten Zahl alle 1er und 0er vertauscht sind. Wie verhalten sich diese beiden Zahlen zueinander? Findet eine Regelmäßigkeit!
\end{enumerate}


\vspace*{10mm}
\printlicense

\printsocials


\end{document}
