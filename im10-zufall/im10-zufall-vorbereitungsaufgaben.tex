\documentclass{cssheet}

%--------------------------------------------------------------------------------------------------------------
% Basic meta data
%--------------------------------------------------------------------------------------------------------------

\title{Dem Zufall auf der Spur}
\author{Prof. Dr. Christian Spannagel}
\date{\today}
\hypersetup{%
    pdfauthor={\theauthor},%
    pdftitle={\thetitle},%
    pdfsubject={Aufgabenblatt Inside Math!},%
    pdfkeywords={insidemath,zufall,stochastik}
}

%--------------------------------------------------------------------------------------------------------------
% document
%--------------------------------------------------------------------------------------------------------------

\begin{document}
\printtitle

\vspace*{10mm}

\textbf{Aufgabe 1 (Zum Rumkugeln):}  In einer Urne sind 10~Kugeln:  vier	rote, drei weiße, zwei blaue und eine grüne Kugel. Man zieht nacheinander zwei Kugeln ohne Zurücklegen. Wie groß ist die Wahrscheinlichkeit für folgende Ereignisse?

\begin{enumerate}[a)]
\item Die erste Kugel ist blau.
\item Beide Kugeln haben die gleiche Farbe.
\item Die zweite Kugel ist rot oder weiß.
\item Die erste Kugel ist nicht weiß, und die zweite ist grün.
\item Mindestens eine Kugel ist rot.
\end{enumerate}


\textbf{Aufgabe 2 (Rain man):} Bei der Prognose des Wetters sind auch immer Wahrscheinlichkeiten im Spiel. Eine Auswertung des Wetterberichts für Neuseeland (der letzten sechs Monate) zeigt, dass für den kommenden Tag zu 54~\% Sonnenschein und zu 46~\% Bewölkung bzw. Regen vorausgesagt wurde. Wurde Sonne prognostiziert, stimmte es zu 80~\%, wurde schlechtes Wetter angemeldet zu 90~\%. Wie viel Prozent schöne Tage gab es?	 

\textbf{Aufgabe 3 (Game over):} Beim Kinderspiel \glqq{}Schere-Stein-Papier\grqq{} müssen zwei Kinder gleichzeitig mit der Hand einen der Begriffe anzeigen. Dabei gilt:
\begin{itemize}
\item Schere schneidet Papier (Schere gewinnt).
\item Papier wickelt den Stein ein (Papier gewinnt).
\item Stein zerschlägt die Schere (Stein gewinnt).
\end{itemize}
Das Spiel ist unentschieden, wenn beide Kinder identische Begriffe zeigen. Stellt das Spiel in einem Baumdiagramm dar. Berechnet dann die Wahrscheinlichkeit für einen unentschiedenen Spielausgang.

\textbf{Aufgabe 4 (Geburtstagsparty):} In einer Schulklasse sind 23~Kinder. Wie wahrscheinlich ist es, dass mindestens zwei am gleichen Tag Geburtstag haben? Wir nehmen mal an, dass in der Klasse keine Zwillinge, Drillinge usw. sind. Außerdem hat keiner am 29.~Februar Geburtstag.  

\newpage
\textbf{Lösungshinweise:} 

Diese Lösungshinweise helfen euch zu schauen, ob ihr richtig gerechnet habt. Falls ihr Fragen zum Lösungsweg habt oder euch die Lösung nicht klar ist, dann postet sie in Discord oder in Moodle!

\begin{itemize}
\item Aufgabe~1: $\frac{1}{5}$, $\frac{2}{9}$, $\frac{7}{10}$, $\frac{1}{15}$, $\frac{2}{3}$
\item Aufgabe~2: Es gab $47,8\%$ schöne Tage.
\item Aufgabe~3: Die Wahrscheinlichkeit für einen unentschiedenen Spielausgang beträgt~$\frac{1}{3}$.
\item Aufgabe~4: Die Wahrscheinlichkeit, dass mindestens 2 Kinder am gleichen Tag Geburtstag haben, beträgt $50,73\%$.
\end{itemize}

\vspace*{3cm}

\printlicense

\printsocials


\end{document}
