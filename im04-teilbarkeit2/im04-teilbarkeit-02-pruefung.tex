\documentclass{cssheet}


%--------------------------------------------------------------------------------------------------------------
% Basic meta data
%--------------------------------------------------------------------------------------------------------------

\title{ggT(Prüfungsaufgaben)}
\author{Prof. Dr. Christian Spannagel}
\date{\today}

%--------------------------------------------------------------------------------------------------------------
% document
%--------------------------------------------------------------------------------------------------------------

\begin{document}
\printtitle

\begin{aufgabe}[WiSe 16/17]
	Bestimmen Sie ggT $(34,21)$ und kgV $(34,21)$.
\end{aufgabe}

\begin{aufgabe}[WiSe 16/17]
	Seien $p$ und $q$ zwei unterschiedliche Primzahlen. Was ist der $\operatorname{ggT}(p, q)$, und was ist das $\operatorname{kgV}(\mathrm{p}, \mathrm{q})$ ? Begründen Sie!
\end{aufgabe}

\begin{aufgabe}[SoSe 17, HT]
	Berechnen Sie den ggT von $256$ und $132$
	\begin{enumerate}
		\item mit Hilfe des Euklidischen Algorithmus.
		\item mit Hilfe der Primfaktorzerlegungen der beiden Zahlen.
	\end{enumerate}
\end{aufgabe}

\begin{aufgabe}[SoSe 17, NT]
	Berechnen Sie den ggT von $264$ und $224$
	\begin{enumerate}
		\item mit Hilfe des Euklidischen Algorithmus.
		\item mit Hilfe der Primfaktorzerlegungen der beiden Zahlen.
	\end{enumerate}
\end{aufgabe}

\begin{aufgabe}[SoSe 22]
	\begin{enumerate}
		\item Bestimmen Sie den ggT von $180$ und $75$ mit dem Euklidischen Algorithmus.
		\item Bestimmen Sie den ggT von $180$ und $75$ über die Primfaktorzerlegung beider Zahlen.
		\item Geben Sie das $\mathrm{kgV}(180,75)$ an.
	\end{enumerate}
\end{aufgabe}

\begin{aufgabe}[SoSe 23]
	Bestimme den ggT der Zahlen $180$ und $75$ mit Hilfe der Primfaktorzerlegungen der beiden Zahlen.
	Bestimme den ggT der Zahlen $180$ und $75$ mit dem Euklidischen Algorithmus.
	Bestimme das kgV der Zahlen $180$ und $75$.
\end{aufgabe}

\begin{aufgabe}[WiSe 23/24]
	Bestimme den größten gemeinsamen Teiler von $2024$ und $231$ mit Hilfe des Euklidischen Algorithmus.
\end{aufgabe}

\begin{aufgabe}[WiSe 24/25]
	Was ist der $\operatorname{ggT}(2025,225)?$
\end{aufgabe}


\vspace*{10mm}
\printlicense

\printsocials



\end{document}
