\documentclass{../cssheet}

%--------------------------------------------------------------------------------------------------------------
% Basic meta data
%--------------------------------------------------------------------------------------------------------------

\title{Teilen ist schön!}
\author{Prof. Dr. Christian Spannagel}
\date{\today}
\hypersetup{%
    pdfauthor={\theauthor},%
    pdftitle={\thetitle},%
    pdfsubject={Aufgabenblatt Inside Math!},%
    pdfkeywords={insidemath}
}


%--------------------------------------------------------------------------------------------------------------
% document
%--------------------------------------------------------------------------------------------------------------

\begin{document}
\printtitle

\vspace*{10mm}

\textbf{Aufgabe 1 (Prima Primzahlen):}  Versucht, möglichst genau den Begriff \emph{Primzahl} zu definieren. Aber Vorsicht: Keine andere Zahl darf unter diese Definition fallen!

\textbf{Aufgabe 2 (Findet alle Teiler):}  Findet alle Teiler der folgenden Zahlen und schreibt sie in Tabellenform übersichtlich auf: $100$, $81$, $17$, $180$.

\textbf{Aufgabe 3 (Ich liebe Hasse!):}  Schreibt die Primfaktorzerlegungen der folgenden Zahlen auf und stellt die Zahlen in Form von Hassediagrammen dar: $100$, $81$, $17$, $180$. 

Vergleicht anschließend die Ergebnisse von Aufgabe~2 und Aufgabe~3. Was stellt ihr fest?

\textbf{Aufgabe 4 (Alle Gummibärchen für mich!):}  
\begin{enumerate}[a)]
\item Wie kann man feststellen, ob eine Zahl durch die Zahlen $2$, $3$, $4$, $5$, $6$, $8$ oder $9$ teilbar ist? Schreibt die Teilbarkeitsregeln auf.  
\item Begründet die Teilbarkeitsregel für die Zahl $3$ mit Hilfe des Gummibärchenmodells.
\end{enumerate}


\vspace*{10mm}
\printlicense

\printsocials

\end{document}
