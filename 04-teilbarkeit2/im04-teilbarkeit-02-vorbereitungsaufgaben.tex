\documentclass{../cssheet}

%--------------------------------------------------------------------------------------------------------------
% Basic meta data
%--------------------------------------------------------------------------------------------------------------

\title{Gegeteh und Kagevau}
\author{Prof. Dr. Christian Spannagel}
\date{\today}
\hypersetup{%
    pdfauthor={\theauthor},%
    pdftitle={\thetitle},%
    pdfsubject={Aufgabenblatt Inside Math!},%
    pdfkeywords={insidemath}
}


%--------------------------------------------------------------------------------------------------------------
% document
%--------------------------------------------------------------------------------------------------------------

\begin{document}
\printtitle 

\vspace*{10mm}

\textbf{Vorbemerkungen:} 
\emph{Es gelten die gleichen Vorbemerkungen wie beim letzten Aufgabenblatt. :-)}

\textbf{Aufgabe 1 (Seid doch mal konstruktiv!):}  Versucht, die Begriffe \emph{größter gemeinsamer Teiler (ggT)} und \emph{kleinstes gemeinsames Vielfaches (kgV)} konstruktiv zu definieren: Was muss man tun, um den ggT bzw. das kgV zweier Zahlen zu bilden? Beschreibt es in Worten und Bildern. 

\textbf{Aufgabe 2 (Hasse 2):} Wie kann man an den Hassediagrammen zweier Zahlen deren ggT und kgV ablesen? Macht euch dies deutlich an den beiden Zahlen $72$ und $108$.

\textbf{Aufgabe 3 (PFZ 2):}  Wie kann man an den Primfaktorzerlegungen zweier Zahlen deren ggT und kgV ablesen? Macht euch dies deutlich an den beiden Zahlen $300$ und $360$.
 
\textbf{Aufgabe 4 (Euklid!):}  
Bestimmt den $ggT(32,125)$ und den $ggT(108,72)$ zunächst mit Hilfe der einzelnen Anwendung der Sätze 1-5, anschließend nochmal mit der -- sehr viel effizierteren -- tabellarischen Form des Euklidischen Algorithmus. 

%\newpage
\vspace*{10mm}
\printlicense

\printsocials

\end{document}
